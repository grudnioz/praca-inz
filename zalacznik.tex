\section{Tytu� rozdzia�u za��cznika}

\subsection{Tytu� podrozdzia�u za��cznika}
\label{sec:podrozdzial_zalacznika}

Jaka� tre��. R�wnanie Kirchhoffa dla obwodu przedstawionego na rys.~\ref{rys:zalacznik:rl}:

\begin{figure}[H]
\begin{center}
dsf
\caption{Obw�d szeregowy $RL$.}\label{rys:zalacznik:rl}
\end{center}
\end{figure}

\noindent ma posta�:

\begin{equation}
    E = u_R + u_L = Ri + L{di\over dt}.
\end{equation}

\noindent W zwyczajowej formie zapisuje si�:

\begin{equation}
    \label{eq:rownanie_w_zalaczniku}
    i(t) =  \left(i_0 - {E\over R}\right)e^{-(R/L) t} + {E\over R}.
\end{equation}

\subsection{Tytu� kolejnego podrozdzia�u za��cznika}
Jaka� tre��...

\section{Tytu� kolejnego rozdzia�u za��cznika}
Jaka� tre��...
